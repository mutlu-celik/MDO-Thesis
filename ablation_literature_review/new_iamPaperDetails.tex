\studentName{Mutlu Çelik}
\advisorName{Prof.~Dr.~Omur Ugur}
% Your Department: 
% Scientific Computing, 
% Financial Mathematics, 
% Cryptography, or 
% Actuarial Sciences
\departmentName{Scientific Computing}
% Type of the Document:
% Preprint # may not work as expected (to-do)
% Qualification in PhD
% Report for Thesis Monitoring Committee
% Term Project
% Draft of a Manuscript
\paperType{Term Report}


\paperTitle{%
	One Dimensional Thermal Analysis Model for Charring Ablative Materials
}

\paperAuthor{%
  \"{O}. U\u{g}ur\footnote{%
    Middle East Technical University, Institute of Applied
    Mathematics, 06800 \c{C}ankaya, Ankara, Turkey. \hfill
  \emph{E-Mail}: \texttt{ougur@metu.edu.tr}},
	% possibly your advisor and co-advisor
  M. Celik\footnote{ Middle East Technical University, Institute of Applied
    Mathematics, 06800 \c{C}ankaya, Ankara, Turkey. \hfill
  \emph{E-Mail}: \texttt{mutlu.celik@metu.edu.tr}},
}

% abstract should not be large to fit in the title/front page
\paperAbstract{%
 This paper presents a one-dimensional model for the analysis of the charring ablative materials used in spacecraft 
thermal protection systems. The numerical method is based on an implicit finite difference formulation of the governing equations 
written for a system of mobile coordinates that accounts for the possible presence of surface recession. The maximum allowable 
operating temperature for the adhesive layer of the junction between the heat shield and the substructure is used as a design 
parameter for determining the minimum heat shield thickness. A case study on the re-entry of the Stardust capsule is presented. 
The model proposed as a useful dimensioning tool for the preliminary design phase of the heat shields of spacecraft entering 
the atmosphere. The model was validated through a survey of the literature related to the dimensioning of thermal shields, but 
based on numeric programs of highly representative industrial standards.
}

\paperKeywords{%
  Thermal protection system, Ablative materials, Thermal analysis
}

\paperDate{July 2021}

